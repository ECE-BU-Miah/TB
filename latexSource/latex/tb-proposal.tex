\section*{List of Abbreviations}
\label{sec:abbreviations}
\addcontentsline{toc}{section}{List of Abbreviations}

\begin{tabular}{ll}
  {\bf PWM}& Pulse Width Modulation\\
\end{tabular}
\section{Background}
\label{sec:background}

Agricultural Spraying machines are used to apply chemical to farmland to increase the productivity of the land.  Fertilizer increases yield by $40\%$ to $60\%.$  If pesticides are not used, yield decreases by $50\%$ to $90\%.$  To maximize these yields, the correct amount of chemical must be applied to the correct area of the plant or soil. Chemical is applied using sprayers which are connected to a pump through plumbing system. A two-way plumbing diagram of an agricultural spraying machine is shown in Figure~\ref{fig:twoWayPlumbingDiagramSrayingMachine}\todo{Illustrate this figure}.
%
\begin{figure}
  \centering
  \includegraphics[width=0.7\textwidth]{figs/img/twoWayPlumbingDiagramSrayingMachine}
  \caption{Two-way plumbing diagram of an agricultural srayping machine [Courtesy of TeeJet Technologies].}
  \label{fig:twoWayPlumbingDiagramSrayingMachine}
\end{figure}



\section{Problem Statement and Objectives}
\label{sec:probl-stat-object}

All sprayers have a pump that forces chemical from a tank through hydraulic plumbing leading to the boom of the sprayer \add{(see Figure~\ref{fig:twoWayPlumbingDiagramSrayingMachine})}.  Flow to the boom can be changed by either changing the speed of the pump or by opening and closing a regulation valve.

Across the boom are a number of nozzles that serve as the exit point for fluid.  These nozzles also have a solenoid that can open or close the valve.  By applying a PWM signal, we can \todo{$\ldots$ continue} When spaying, the farmer is limited to how fast he/she can drive and apply the correct amount of chemical without increase. In a sprayer application, we have two separate discrete-time feedback control systems to solve these problems.

\begin{enumerate}
\item One system controls to a desired flow rate\todo{Add unit}.  This system ensures the correct amount of fluid is applied to the field.  Target flow is calculated by the prescription for the field and the speed of the vehicle. 

\item A second system controls to a constant pressure across the boom of a sprayer.  Droplet size directly correlates to pressure and can be determined from the nozzle’s datasheet.  This ensures proper application of the fluid and prevents drift due to atmospheric conditions.  

\end{enumerate}

Flow rate controllers, pressure controllers, and sprayers can all be manufactured by  different companies.  This makes it very difficult to come up 
with a design that will \replace{preform}{perform} well across different
scenarios.  \add{Figure~\ref{fig:twoInputTwoOutputDT-ControlSystem} shows the
  block diagram of a two-input two-output discrete-time control system block
  diagram, where  $D_2(z),$ $G(z),$ $x_2^{[d]},$ and $x_2$ are assumed to be
  unknown.} %
%

\begin{figure}
  \centering
  \begin{tikzpicture}    
      \tikzstyle{every node} = [font=\footnotesize]
      \tikzstyle{block} = [draw, rectangle, fill=blue!15, rounded corners, minimum height=0.5 cm, minimum width=1.0 cm]
      \tikzstyle{sum} = [draw, circle, fill=blue!15];
      \tikzstyle{pinstyle} = [pin edge={to-,thin,black}]
      % Place nodes
      \node[sum](sum){$\sum$};
      \node[block, text width=1.5 cm, text centered, right of = sum, node distance=3 cm](D1){Controller $D_1(z)$};
      \node[block, text width=1.5 cm, text centered, below of = D1, node distance=2 cm](D2){Controller $D_2(z)$};
      \node[sum, left of = D2, node distance=2.0 cm](sum3){$\sum$};
      \node[block, text width=1 cm, text centered, right of = D2, node distance=4.5 cm](G){Plant $G(z)$};      
      \node[block, text width=1.5 cm, text centered, below right of = D2, node distance=3.0 cm](H2){Flow meter $H_2(z)$};
      \node[block, text width=1.5 cm, text centered, below of = H2, node distance=2.0 cm](H1){Pressure sensor $H_1(z)$};
      \node[sum, right of = H2, node distance=2 cm](sum2){$\sum$};
      \node[sum, right of = H1, node distance=2 cm](sum1){$\sum$};      

      % Connections 
      \draw[latex-]
      (sum.west)-- node[midway,above]{$x_1^{[d]}$}++(-2*\smgrid,0)node[anchor=east]{Target pressure};
      \draw
      (sum.-165)node[left]{$+$};
      \draw[latex-]
      (sum3.west)-- node[midway,above]{$x_2^{[d]}$}++(-4*\smgrid,0)node[anchor=east]{Target flow};
      \draw
      (sum3.-165)node[left]{$+$};
      
      \draw[-latex]
      (sum.east) -- node[midway,above]{$e_1$}(D1.west);
      \draw[-latex]
      (sum3.east) -- node[midway,above]{$e_2$}(D2.west);
      \draw[-latex]
      (D1.east) -| node[near start,above]{$u_1$} (G.north);
      \draw[-latex]
      (D2.15) --++(2*\smgrid,0)|- node[near end,above]{$u_2$}(G.west);
      \draw[-latex]
      (D2.-15) -|node[very near end,right]{$u_3$}++(2*\smgrid,-1.5*\smgrid)-| (G.south);

      \draw[-latex]
      (G.20) -- ++(3*\smgrid,0)node[above,right]{$Y_1(z)$};
      \draw[-latex]
      (G.-20) -- ++(3*\smgrid,0)node[below,right]{$Y_2(z)$};
      \draw[-latex]
      ($(G.20) + (1.5*\smgrid,0)$)node[fill,circle,inner sep=1.5pt]{} |- (sum1.east)node[very near end,below]{$+$};
      \draw[-latex]
      ($(G.-20) + (\smgrid,0)$)node[fill,circle,inner sep=1.5pt]{} |- (sum2.east)node[very near end,below]{$+$};

      \draw[-latex]
      (sum2.west) -- (H2.east);
      \draw[-latex]
      (H2.west) -| node[near start,above]{$x_2$} (sum3.south)node[anchor=north east]{$-$}; 
      
      \draw[-latex]
      (sum1.west) -- (H1.east);
      \draw[-latex]
      (H1.west) -|node[near start,above]{$x_1$} (sum.south)node[anchor=north east]{$-$}; 

      \draw[latex-]
      (sum2.south) -- ++(0,-\smgrid)node[below]{$w_2$};
      \draw
      ($(sum2.-90)+(0.05*\smgrid,-0.2*\smgrid)$)node[below,right]{$+$};
      \draw[latex-]
      (sum1.south) -- ++(0,-\smgrid)node[below]{$w_1$};
      \draw
      ($(sum1.-90)+(0.05*\smgrid,-0.2*\smgrid)$)node[below,right]{$+$};
      
      % Layers
      \begin{pgfonlayer}{background}
        \filldraw[fill=yellow!10,draw = blue,dashed,very thick, rounded corners]
        ($(sum.north west) + (-1.9*\smgrid,2*\smgrid)$) rectangle ($(sum1.south east)+(3.5*\smgrid,-2.2*\smgrid)$);
      \end{pgfonlayer}
      \draw
      ($(D1.north)+(\smgrid,\smgrid)$)node[]{\textcolor{blue}{{\bf Discrete-time control system}}};
    \end{tikzpicture}  
  \caption[Two-input, two-output discrete-time control system block diagram.]{Two-input, two-output discrete-time control system block diagram.}
  \label{fig:twoInputTwoOutputDT-ControlSystem}
\end{figure}
%
\begin{table}
  \centering
  \caption{Description of signals used in the control system block diagram shown in Figure~\ref{fig:twoInputTwoOutputDT-ControlSystem}.}
  \label{tab:signalDescription}
  \begin{tabular}{ll}
    \toprule[1.5pt]
    Signal& Description\\
    \toprule
    $x_1^{[d]}$ & Target pressure\\
    $x_2^{[d]}$ & Target flow\\
    $x_1$ & Actual pressure\\
    $x_2$ & Actual flow\\
    $D_1(z)$ & Pressure controller \add{transfer function}\\
    $D_2(z)$ & Flow controller \add{transfer function}\\
    $u_1$ & Duty cycle to solenoids (limited between $0$ and $100$)\\
    $u_2$ & Open/close signal for regulating valve\\
    $u_3$ & Pump speed (optional)\\
    $G(z)$ & Plant model\\
    $H_1(z)$ & Pressure sensor model (transfer function)\\
    $H_2(z)$ & Flow meter model (transfer function)\\
    $w_1$ & Pressure sensor noise\\
    $w_2$ & Flow meter noise\\
    \bottomrule[1.5pt]
  \end{tabular}
\end{table}
%

{\bf Problem statement:} Develop a control system that can adapt to different plant models and \comment{rate controllers.}{Do you mean flow rate and target pressure controllers?}

{\bf Outcomes:}

\textit{TeeJet} expects design of a  control system that can adapt to: %
\begin{enumerate}
\item Different plant models
  
\item Different rate controllers 
\end{enumerate}

\textit{Student experience}  are expected to learn the following items that pertain to proposed control system:
\begin{enumerate}
\item Design the control system that is ready for implementation
  
\item Conduct computer simulations using MATLAB and Simulink
  
\item Validation and testing using real-time embedded system 
  

\end{enumerate}


\section{Solution Approach}
\label{sec:solutionApproach}

Model-Free Reinforcement Learning Control Approach


\section{Deliverables}
\label{sec:deliverables}

TBD

\section{Timeline and Milestones}
\label{sec:timeline}



% %
\begin{figure}
  % \begin{sidewaysfigure}
    \centering
    \begin{ganttchart}[hgrid,
      vgrid={*8{black, dotted},*8{black,dotted},*8{black, dotted},*8{black,dotted}},
      title/.style={fill=gray!15,draw=black},
    group/.append style={draw=black, fill=yellow!50},
    group left shift = 0,
    group right shift = 0,
    x unit=.5cm,  % Controls the x-axis grid width of the chart
    y unit title=.8cm,
    y unit chart=.6cm,
    milestone label font=\tiny,
    bar label font=\small, 
    group label font=\normalsize,
    bar/.append style={draw=black,fill=blue},
    bar left shift= 0,
    bar right shift= 0,
    bar height = .7]{1}{20}  % Total number of weeks, from 1 to 20, for example 
    \gantttitle{2020}{20}  % Number of weeks in 2020 
    % \gantttitle{2021}{16}    % Number of weeks in 2021 
    \\
    \gantttitle{July}{4}
    \gantttitle{Aug}{4}
    \gantttitle{Sep}{4}
    \gantttitle{Oct}{4}
    \gantttitle{Nov}{4}\\
    \ganttgroup[progress=25]{{\bf Problem description}}{1}{4}\\
    \ganttbar{Problem analysis}{1}{2}\\
    \ganttbar{Deliverables}{2}{4}\\
%    
    \ganttgroup[progress=25]{{\bf Solution approach}}{2}{8}\\
    \ganttbar{Model-free analysis}{3}{5}\\
    \ganttbar{Reinforcement learning setup}{4}{8}\\
    \ganttgroup[progress=0]{{\bf Implementation}}{8}{16}\\
    \ganttbar{Matlab simulation}{8}{10}\\
    \ganttbar{Simulink implementation}{8}{12}\\
    \ganttbar{Embedded system implementation}{10}{16}\\
    \ganttgroup[progress=0]{{\bf Publication/report}}{16}{20}\\
    \ganttbar{Conference/journal paper}{16}{18}\\
    \ganttbar{Report/workshop/presentation}{16}{20}
  \end{ganttchart}
\caption{Gantt chart showing the project activities from July 2020 to November 2020.}
\label{fig:gantt1}
% \end{sidewaysfigure}
\end{figure}

  
% \end{itemize}
% %
% \includepdf[pagecommand={\thispagestyle{fancy}}]{supportingDocs/seedGrantBEMS-AmerenV1.pdf}      

% \subsection{Biographical Sketch of PI}
% PI's biographical sketch is attached in the next page. 
% \includepdf[pages=-,pagecommand={\pagestyle{fancy}}]{supportingDocs/bioSketchMiah.pdf}



%%% Local Variables:
%%% mode: latex
%%% TeX-master: "../mainProposal"
%%% End:
